\input luatexbase.sty
\directlua{%
local linebreaker = require "linebreaker"
linebreaker.max_cycles = 90
luatexbase.add_to_callback("linebreak_filter", linebreaker.linebreak, "new linebreak callback")
%luatexbase.add_to_callback("post_linebreak_filter", linebreaker.traverse, "xxx")
}

\hsize=150pt
The example document given below creates two pages by using Lua code alone. You will learn how to access TeX's boxes and counters from the Lua side, shipout a page into the PDF file, create horizontal and vertical boxes (hbox and vbox), create new nodes and manipulate the nodes links structure. The example covers the following node types: rule, whatsit, vlist, hlist and action.

In the example code we use black squares as the contents of the pages and not normal text, because character handling takes more code and will be covered later on.

Save the following file into myprogram.lua and use the TeX stub from above (or get the source here). 

% sdahfjhasjfdhksdafdsfasfjhasjhfasdfhjhsdahh

\bye
