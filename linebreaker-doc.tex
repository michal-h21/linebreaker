\documentclass{ltxdoc}


\usepackage[english]{babel}
\usepackage{hyperref}
\newcommand\authormail[1]{\footnote{\textless\url{#1}\textgreater}}
\ifdefined\HCode
\renewcommand\authormail[1]{\space\textless\Link[#1]{}{}#1\EndLink\textgreater}
\fi

\usepackage{listings}
\usepackage{fontspec}
\setmainfont{TeX Gyre Schola}
% \setmonofont[Scale=MatchLowercase]{Inconsolatazi4}
\IfFontExistsTF{Noto Sans Mono Regular}{%
    \setmonofont[Scale=MatchLowercase]{Noto Sans Mono Regular}
}{\setmonofont{NotoMono-Regular.ttf}}
\usepackage{upquote}

\usepackage{microtype}

\title{The \texttt{Linebreaker} package}
\author{Michal Hoftich\authormail{michal.h21@gmail.com}}
\date{Version \version\\\gitdate}
\begin{document}
\maketitle
\tableofcontents

\section{Introduction}

Purpose of this experimental package is to prevent paragraph overflow in
LuaTeX. Sometimes, TeX cannot find good points for line-breaking. This results
in lines where part of words stick out of the paragraph shape. Linebreaker
tries to prevent that using repeated execution of the line-breaking algorithm
with different values for parameters which are taken into account in this
process.


\section{Usage}


\begin{lstlisting}{latex}
\usepackage{linebreaker}
\end{lstlisting}


\subsection{Example}


\section{Some background}

This repository contains experimental line-breaking callback for LuaTeX engine (
it should work with any format). Motivation for this was a
\href{http://tex.stackexchange.com/q/200989/2891}{question} by Frank Mittelbach on
TeX.sx. His idea is to rewrite \TeX\ paragraph building algorithm in Lua, in
order to support river detection and similar tasks, unsupported by standard TeX
line-breaking algorithm.

As complete rewrite of line-breaking algorithm seems to be huge task, I tried
different approach. LuaTeX provides several callbacks for working with node lists. 
There callbacks are called when some actions on the node lists happens. For
example ligaturing, kerning, before line-breaking, after line-breaking and
callback for doing the line-breaking. There is a `tex.linebreak` function, which takes
node list and table with TeX parameters (`lineskip`, `baselineskip`, `tolerance`,
etc.) New node list with lines broken into horizontal boxes is returned by this
function.

My idea is to process this returned node list, detect problems and call
`tex.linebreak` with different parameters if problems were detected. At the
moment, overflow box detection works somehow, river detection is a proof of
concept and it needs further corrections.


\end{document}

