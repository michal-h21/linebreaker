\documentclass{l3doc}


% \usepackage[latin,english]{babel}=15pt
\usepackage{lipsum}
\usepackage{linebreaker}
\usepackage{lua-widow-control}
\linebreakersetup{debug}
\usepackage{hyperref}
\newcommand\authormail[1]{\footnote{\textless\url{#1}\textgreater}}
\ifdefined\HCode
\renewcommand\authormail[1]{\space\textless\Link[#1]{}{}#1\EndLink\textgreater}
\fi

\ifdefined\gitdate\else\def\gitdate{Undefined}\fi
\ifdefined\version\else\def\version{Undefined}\fi

\usepackage{listings}
\usepackage{fontspec}
\setmainfont{TeX Gyre Schola}
% \setmonofont[Scale=MatchLowercase]{Inconsolatazi4}
\IfFontExistsTF{Noto Sans Mono Regular}{%
    \setmonofont[Scale=MatchLowercase]{Noto Sans Mono Regular}
}{\setmonofont{NotoMono-Regular.ttf}}
\usepackage{upquote}

\usepackage{microtype}

\newcommand\testbox[1]{%
  \parbox{150pt}{%
    \parindent=15pt%
    \tolerance=1%
    \pretolerance=1%
    #1
  }%
}

\newcommand\printtest[1]{%
  \linebreakerdisable%
  \noindent\testbox{%
    #1
    \par\medskip\noindent\hfill\textbf{Without Linebreaker}\hfill\null
  }%
  \linebreakerenable%
  \hfill%
  \testbox{%
    #1
    \par\medskip\noindent\hfill\textbf{With Linebreaker}\hfill\null
  }%
}

\title{The \texttt{Linebreaker} package}
\author{Michal Hoftich\authormail{michal.h21@gmail.com}}
\date{Version \version\\\gitdate}
\begin{document}
\maketitle
\tableofcontents

\section{Introduction}

This package tries to prevent overflow lines in paragraphs or boxes.
It changes the Lua\TeX's \verb|linebreak| callback, and it re-typesets the paragraph 
with increased values of \cmd{\tolerance} and \cmd{\emergencystretch}
until the overflow doesn't happen. If that doesn't help, it chooses the solution
with the lowest badness.


The advantage of this approach is that paragraphs that have not overflowed are
typeset with default parameters. These are changed only for problematic
paragraphs.

The code is experimental, and you may find bugs or clashes with
other packages. You can send bug reports to the package's repository on 
Github\footnote{\url{https://github.com/michal-h21/linebreaker}}.



% \noindent\begin{minipage}{220pt}
 \def\testtext{%
The example document given below creates two pages by using Lua code alone. You
will learn how to access TeX's boxes and counters from the Lua side, shipout a
page into the PDF file, create horizontal and vertical boxes (hbox and vbox),
create new nodes and manipulate the nodes links structure. The example covers
the following node types: rule, whatsit, vlist, hlist and action.
 }



\begin{figure}
\printtest\testtext%
\caption{Example of Linebreaker's effect}
\end{figure}

% \end{minipage}
\newpage
\section{Usage}

There is only \LaTeX\ package at the moment. Con\TeX t and Plain \TeX\ are not supported.
You can enable the overflow paragraph handling by loading of the Linebreaker package:

% \begin{lstlisting}{latex}
% \usepackage[options]{linebreaker}
\begin{center}
\cmd{\usepackage}\verb|{linebreaker}|
\end{center}
% \end{lstlisting}

\subsection{Enable and disable Linebreaker}

The package owerflow handling is enabled by default. You can disable it and then re-enable using the following commands:

\begin{function}{\linebreakerdisable}
Disable line-breaking processing. \LaTeX\ will typeset the following paragraphs with the default values for line-breaking.
\end{function}

\begin{function}{\linebreakerenable}
Re-enable line-breaking processing after it was disabled by \cmd{\linebreakerdisable}.
\end{function}

\subsection{Change of Linebreaker parameters}

\begin{function}{\linebreakersetup}
  Change settings of the line-breaking algorithm.
  Usage: \cmd{\linebreakersetup}\Arg{options}


\end{function}

\subsubsection{Available options for the \cmd{\linebreakersetup} command}

\begin{function}{maxcycles}
Number of attempts to re-typeset the paragraph.
\end{function}

\begin{function}{maxemergencystretch}
Maximal allowed value of \verb|\emergencystretch|.
\end{function}

\begin{function}{maxtolerance}
Maximal allowed value of \verb|tolerance|.
\end{function}

\begin{function}{cubic}
Use cubic method for the \verb|tolerance| calculation. By default, \verb|tolerance| is calculated
using a linear method, with constant steps. With the cubic method, each step is larger than the previous.
\end{function}

\begin{function}{debug}
Print debugging info to the terminal output.
\end{function}

\subsubsection{Example of \cmd{\linebreakersetup} use}

\begin{lstlisting}{latex}
\linebreakersetup{
  maxtolerance=90,
  maxemergencystretch=1em,
  maxcycles=4
}
\end{lstlisting}



\section{Historical background}

The motivation to create this package was a 
question\footnote{\url{http://tex.stackexchange.com/q/200989/2891}} by Frank Mittelbach on
TeX.SE. His idea was to rewrite TeX’s paragraph-building algorithm in Lua to
allow detection of rivers and similar tasks unsupported by the standard TeX
line-breaking algorithm.

As a complete rewrite of the line-breaking algorithm seemed too complicated, 
I tried a different approach. Lua\TeX\ provides  callbacks for working with node lists. 
It calls these callbacks when actions on the node lists happen, such as 
ligaturing, kerning, before line-breaking, after line-breaking, and 
callback that  handles the line-breaking process. 

There is a \verb|tex.linebreak| function, which takes
node list and table with \TeX\ parameters (like \verb|lineskip|, \verb|baselineskip|, \verb|tolerance|,
etc.). It returns a new node list, with lines broken into horizontal boxes.

My idea is to process this returned node list, detect problems and call
`tex.linebreak` with different parameters if lines overflow. At the
moment, overflow box detection works in most cases, but river detection is unusable, and it needs further corrections.

\section{License}

Permission is granted to copy, distribute and/or modify this software
under the terms of the LaTeX Project Public License, version 1.3.

\section{Changes}

\begin{description}
  \item[v0.1c, 2023-???]\hfill
    \begin{itemize}
      \item Pass default parameters to all calls of \verb|tex.linebreak|, in order to support Bidi text in all instances.
    \end{itemize}
  \item[v0.1b, 2023-03-08]\hfill
    \begin{itemize}
      \item Set maximal value of tolerance to 8189, bigger value doesn't have any effect.
      \item Added \verb|cubic| method for calculating tolerance.
      \item Explicitly set the \verb|tex.linebreak| parameters, to support paragraph direction and other paragraph variables.
    \end{itemize}
  \item[v0.1a, 2022-03-12]\hfill

    \begin{itemize}
      \item Fixed fatal error in the function that calculates width of last lines in paragraphs.
    \end{itemize}

  \item[v0.1, 2022-02-19]\hfill

    \begin{itemize}
      \item Initial version
    \end{itemize}
\end{description}

\end{document}

