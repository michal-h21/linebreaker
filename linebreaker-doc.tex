\documentclass{l3doc}


% \usepackage[latin,english]{babel}=15pt
\usepackage{lipsum}
\usepackage{linebreaker}
\linebreakersetup{debug}
\usepackage{hyperref}
\newcommand\authormail[1]{\footnote{\textless\url{#1}\textgreater}}
\ifdefined\HCode
\renewcommand\authormail[1]{\space\textless\Link[#1]{}{}#1\EndLink\textgreater}
\fi

\ifdefined\gitdate\else\def\gitdate{Undefined}\fi
\ifdefined\version\else\def\version{Undefined}\fi

\usepackage{listings}
\usepackage{fontspec}
\setmainfont{TeX Gyre Schola}
% \setmonofont[Scale=MatchLowercase]{Inconsolatazi4}
\IfFontExistsTF{Noto Sans Mono Regular}{%
    \setmonofont[Scale=MatchLowercase]{Noto Sans Mono Regular}
}{\setmonofont{NotoMono-Regular.ttf}}
\usepackage{upquote}

\usepackage{microtype}

\newcommand\testbox[1]{%
  \parbox{150pt}{%
    \parindent=15pt%
    \tolerance=1%
    \pretolerance=1%
    #1
  }%
}

\newcommand\printtest[1]{%
  \linebreakerdisable%
  \noindent\testbox{%
    #1
  }%
  \linebreakerenable%
  \hfill%
  \testbox{%
    #1
  }%
}

\title{The \texttt{Linebreaker} package}
\author{Michal Hoftich\authormail{michal.h21@gmail.com}}
\date{Version \version\\\gitdate}
\begin{document}
\maketitle
\tableofcontents

\section{Introduction}

This package tries to prevent overflow lines in paragraphs or boxes.
It changes the Lua\TeX's \verb|linebreak| callback, and it re-typesets the paragraph 
with increased values of \verb|\tolerance| and \cmd{\emergencystretch}
until the overflow doesn't happen. If that doesn't help, it chooses the solution
with the lowest badness.

The code is experimental, and you may find bugs or clashes with
other packages. Please send a bug report to the package's repository on 
Github\footnote{\url{https://github.com/michal-h21/linebreaking}}


\bigskip
\noindent\textbf{Example of use:}
\bigskip

% \noindent\begin{minipage}{220pt}
 \def\testtext{%
The example document given below creates two pages by using Lua code alone. You
will learn how to access TeX's boxes and counters from the Lua side, shipout a
page into the PDF file, create horizontal and vertical boxes (hbox and vbox),
create new nodes and manipulate the nodes links structure. The example covers
the following node types: rule, whatsit, vlist, hlist and action.
 }



\printtest\testtext%

% \end{minipage}
\section{Usage}

% \begin{lstlisting}{latex}
% \usepackage[options]{linebreaker}
\cmd{\usepackage}\oarg{options}\verb|{linebreaker}|
% \end{lstlisting}

\subsection{Provided commands}

\begin{function}{\linebreakerdisable}
disable line-breaking processing. \LaTeX\ will typeset the following paragraphs with the default values for line-breaking.
\end{function}

\begin{function}{\linebreakerenable}
re-enable line-breaking processing after it was disabled by \cmd{\linebreakerdisable}.
\end{function}

\begin{function}{\linebreakersetup}
  Change settings of the line-breaking algorithm.
  

  \noindent Usage: \cmd{\linebreakersetup}\Arg{options}

  \noindent Available options are:
\end{function}

\begin{function}{maxcycles}
number of attempts to re-typeset the paragraph.
\end{function}

\begin{function}{maxemergencystretch}
maximal allowed value of \verb|\emergencystretch|.
\end{function}

\begin{function}{maxtolerance}
maximal allowed value of \verb|tolerance|.
\end{function}

\begin{function}{debug}
print debugging info to the terminal output.
\end{function}


\begin{lstlisting}{latex}
\linebreakersetup{
  maxtolerance=90,
  maxemergencystretch=1em,
  maxcycles=4
  }
\end{lstlisting}



\section{Historical background}

The motivation to create this package was a 
\href{http://tex.stackexchange.com/q/200989/2891}{question} by Frank Mittelbach on
TeX.sx. His idea was to rewrite TeX’s paragraph-building algorithm in Lua to
allow detection of rivers and similar tasks unsupported by the standard TeX
line-breaking algorithm.

As a complete rewrite of the line-breaking algorithm seemed too complicated, 
I tried a different approach. Lua\TeX\ provides  callbacks for working with node lists. 
It calls these callbacks when actions on the node lists happen, such as 
ligaturing, kerning, before line-breaking, after line-breaking, and 
callback that  handles the line-breaking process. 

There is a \verb|tex.linebreak| function, which takes
node list and table with \TeX\ parameters (like \verb|lineskip|, \verb|baselineskip|, \verb|tolerance|,
etc.). It returns a new node list, with lines broken into horizontal boxes.

My idea is to process this returned node list, detect problems and call
`tex.linebreak` with different parameters if lines overflow. At the
moment, overflow box detection works in most cases, but river detection is unusable, and it needs further corrections.

\section{Changes}

\begin{description}
  \item[v0.1]
    \begin{itemize}
      \item Initial version
    \end{itemize}
\end{description}

\end{document}

